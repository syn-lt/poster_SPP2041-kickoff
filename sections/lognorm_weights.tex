\columnbreak
\section*{\LARGE Log-normal distributions of synaptic weights in noise driven networks}

In cortical circuits the distribution of synaptic weights has been repeatedly reported to be log-normal \cite{Song2005}. A number of computational models address the how such a weight distribution might arise from network dynamics. Here we test a hypothesis that log-normal weight distributions emerge from network dynamics in which LTP dominates and in which synaptic scaling takes the role of keeping weights from growing too large.

\vspace{-0.4cm}

\section*{Network model}

The network considered here consist of $N$ randomly connected ($p=0.1$) conductance based leaky integrate-and-fire neurons. The differential equation for a neuron's membrane voltage $V$ is
%
\begin{align}
 \tau_m\, \frac{dV}{dt}  = E_l - V + g_e \,(E_e - V) +  \xi_{\mathrm{ext}}(t). \label{eq:mem}
\end{align}
%
When synaptic transmission is active ($\Theta_{\text{trans}}=1$), the conductance $g_e$ is increased by $w_i$ at times $t_i^k$ of spikes from connected neurons with index $i$,
%
\begin{align}
 \frac{d ge}{dt} = - \frac{g_e}{\tau_e} +  \Theta_{\text{trans}}\, \sum_i w_i \sum_{k} \delta(t-t_i^k)
\end{align}
%
Spike-timing dependent plasticity is implemented in the form of long-term potentiation (LTP). The weight $w$ changes for $\Delta t = t_{\text{post}}-t_{\text{pre}} > 0$ as
\begin{align}
 \Delta w =  A_{\text{LTP}} \exp\left(\frac{\Delta t}{\tau_{\text{LTP}}}\right).
\end{align}
%
Multiplicative synaptic scaling acts homeostatic mechanism and prevents weights from unbounded growth. The incoming weights $w_j$ to a given neuron are scaled as
%
\begin{align}
 w_j \to w_j \left(1+ \eta_{\text{SN}} \left( \frac{W_{\text{target}}}{\sum_k w_k} -1\right) \right), \label{eq:sc}
\end{align}
%
where $\sum_k w_k$ sums over all incoming weights and $W_{\text{target}}$ is a set target for the sum.

\vspace{-0.4cm}

\section*{Results}

At first we disabled synaptic transmission $\Theta_{\text{trans}}=0$ resulting in network dynamics that are exclusively driven by the membrane noise $\xi_{\text{ext}}(t)$. We found that log-normal like distributions in synaptic weights naturally emerge from the interplay of additive changes through LTP and the scaling of weights through synaptic normalization, when roughly
%
\begin{align}
 A_{\text{LTP}} =  \frac{W_{\text{target}}}{p\, N}, \label{eq:rel}
\end{align}
that is when the sum of $A_{\text{LTP}}$ for all incoming connections roughly matches the target weight (Fig.~\ref{fig:lognorm_LTP}).

\vspace{0.8cm}
\begin{overpic}[width=.49\columnwidth]%
  % 110, 133
  {figures/syn_lnt_gnw_e07036_16.pdf}
  \put(11,60){\normalfont \textbf{A}}
\end{overpic}
\begin{overpic}[width=.49\columnwidth]%
  % 110, 133
  {figures/syn_lnt_gnw_e07036_17.pdf}
  \put(11,60){\normalfont \textbf{B}}
\end{overpic}
\captionof{figure}{In a completely noise-driven network ($N=400$, $p=0.1$, $\Theta_{\text{trans}}=0$), log-normal like distribution emerge when the relation \eqref{eq:rel} roughly holds. \label{fig:lognorm_LTP}}
\vspace{2.8cm}

We tested relation \eqref{eq:rel} by, for example, increasing $p$. In this case, to obtain good fits for a log-normal distribution in synaptic weights, $A_{\text{LTP}}$ had to be decreased in accordance with the equation. We found that the rate of synaptic scaling $\eta_{\text{SN}}$ did not significantly affect the shape of the weight distribution.
\medskip

In the next step, we analyzed a network with an excitatory an inhibitory population, supplementing equations \eqref{eq:mem}-\eqref{eq:sc} in the typical way and additional adding long-term depression (LTD) for $\Delta t < 0$. First results here indicate that when LTP dominates over LTD ($A_{\text{LTP}} = 10 A_{\text{LTD}}$)a reduction of membrane noise lets log-normal like distribution in synaptic weights emerge more clearly (Fig.~\ref{fig:loglorm_rec}). Correlated activity might thus play a crucial in forming the long tailed distributions of weights observed in the cortical networks.


\vspace{1cm}
\begin{overpic}[width=.49\columnwidth]%
  % 110, 133
  {figures/syn_lnt_gnw_b20efe_0_highmem.pdf}
  \put(11,60){\normalfont \textbf{A}}
\end{overpic}
\begin{overpic}[width=.49\columnwidth]%
  % 110, 133
  {figures/syn_lnt_gnw_b20efe_0_lowmem.pdf}
  \put(11,60){\normalfont \textbf{B}}
  \end{overpic}
\captionof{figure}{In a network with recurrent dynamics, reduction in membrane noise favours log-normal like weight distributions \label{fig:loglorm_rec}}






